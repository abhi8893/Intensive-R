\documentclass[]{article}
\usepackage{lmodern}
\usepackage{amssymb,amsmath}
\usepackage{ifxetex,ifluatex}
\usepackage{fixltx2e} % provides \textsubscript
\ifnum 0\ifxetex 1\fi\ifluatex 1\fi=0 % if pdftex
  \usepackage[T1]{fontenc}
  \usepackage[utf8]{inputenc}
\else % if luatex or xelatex
  \ifxetex
    \usepackage{mathspec}
  \else
    \usepackage{fontspec}
  \fi
  \defaultfontfeatures{Ligatures=TeX,Scale=MatchLowercase}
\fi
% use upquote if available, for straight quotes in verbatim environments
\IfFileExists{upquote.sty}{\usepackage{upquote}}{}
% use microtype if available
\IfFileExists{microtype.sty}{%
\usepackage{microtype}
\UseMicrotypeSet[protrusion]{basicmath} % disable protrusion for tt fonts
}{}
\usepackage[margin=1in]{geometry}
\usepackage{hyperref}
\hypersetup{unicode=true,
            pdftitle={R-Practice},
            pdfborder={0 0 0},
            breaklinks=true}
\urlstyle{same}  % don't use monospace font for urls
\usepackage{color}
\usepackage{fancyvrb}
\newcommand{\VerbBar}{|}
\newcommand{\VERB}{\Verb[commandchars=\\\{\}]}
\DefineVerbatimEnvironment{Highlighting}{Verbatim}{commandchars=\\\{\}}
% Add ',fontsize=\small' for more characters per line
\usepackage{framed}
\definecolor{shadecolor}{RGB}{248,248,248}
\newenvironment{Shaded}{\begin{snugshade}}{\end{snugshade}}
\newcommand{\AlertTok}[1]{\textcolor[rgb]{0.94,0.16,0.16}{#1}}
\newcommand{\AnnotationTok}[1]{\textcolor[rgb]{0.56,0.35,0.01}{\textbf{\textit{#1}}}}
\newcommand{\AttributeTok}[1]{\textcolor[rgb]{0.77,0.63,0.00}{#1}}
\newcommand{\BaseNTok}[1]{\textcolor[rgb]{0.00,0.00,0.81}{#1}}
\newcommand{\BuiltInTok}[1]{#1}
\newcommand{\CharTok}[1]{\textcolor[rgb]{0.31,0.60,0.02}{#1}}
\newcommand{\CommentTok}[1]{\textcolor[rgb]{0.56,0.35,0.01}{\textit{#1}}}
\newcommand{\CommentVarTok}[1]{\textcolor[rgb]{0.56,0.35,0.01}{\textbf{\textit{#1}}}}
\newcommand{\ConstantTok}[1]{\textcolor[rgb]{0.00,0.00,0.00}{#1}}
\newcommand{\ControlFlowTok}[1]{\textcolor[rgb]{0.13,0.29,0.53}{\textbf{#1}}}
\newcommand{\DataTypeTok}[1]{\textcolor[rgb]{0.13,0.29,0.53}{#1}}
\newcommand{\DecValTok}[1]{\textcolor[rgb]{0.00,0.00,0.81}{#1}}
\newcommand{\DocumentationTok}[1]{\textcolor[rgb]{0.56,0.35,0.01}{\textbf{\textit{#1}}}}
\newcommand{\ErrorTok}[1]{\textcolor[rgb]{0.64,0.00,0.00}{\textbf{#1}}}
\newcommand{\ExtensionTok}[1]{#1}
\newcommand{\FloatTok}[1]{\textcolor[rgb]{0.00,0.00,0.81}{#1}}
\newcommand{\FunctionTok}[1]{\textcolor[rgb]{0.00,0.00,0.00}{#1}}
\newcommand{\ImportTok}[1]{#1}
\newcommand{\InformationTok}[1]{\textcolor[rgb]{0.56,0.35,0.01}{\textbf{\textit{#1}}}}
\newcommand{\KeywordTok}[1]{\textcolor[rgb]{0.13,0.29,0.53}{\textbf{#1}}}
\newcommand{\NormalTok}[1]{#1}
\newcommand{\OperatorTok}[1]{\textcolor[rgb]{0.81,0.36,0.00}{\textbf{#1}}}
\newcommand{\OtherTok}[1]{\textcolor[rgb]{0.56,0.35,0.01}{#1}}
\newcommand{\PreprocessorTok}[1]{\textcolor[rgb]{0.56,0.35,0.01}{\textit{#1}}}
\newcommand{\RegionMarkerTok}[1]{#1}
\newcommand{\SpecialCharTok}[1]{\textcolor[rgb]{0.00,0.00,0.00}{#1}}
\newcommand{\SpecialStringTok}[1]{\textcolor[rgb]{0.31,0.60,0.02}{#1}}
\newcommand{\StringTok}[1]{\textcolor[rgb]{0.31,0.60,0.02}{#1}}
\newcommand{\VariableTok}[1]{\textcolor[rgb]{0.00,0.00,0.00}{#1}}
\newcommand{\VerbatimStringTok}[1]{\textcolor[rgb]{0.31,0.60,0.02}{#1}}
\newcommand{\WarningTok}[1]{\textcolor[rgb]{0.56,0.35,0.01}{\textbf{\textit{#1}}}}
\usepackage{graphicx,grffile}
\makeatletter
\def\maxwidth{\ifdim\Gin@nat@width>\linewidth\linewidth\else\Gin@nat@width\fi}
\def\maxheight{\ifdim\Gin@nat@height>\textheight\textheight\else\Gin@nat@height\fi}
\makeatother
% Scale images if necessary, so that they will not overflow the page
% margins by default, and it is still possible to overwrite the defaults
% using explicit options in \includegraphics[width, height, ...]{}
\setkeys{Gin}{width=\maxwidth,height=\maxheight,keepaspectratio}
\IfFileExists{parskip.sty}{%
\usepackage{parskip}
}{% else
\setlength{\parindent}{0pt}
\setlength{\parskip}{6pt plus 2pt minus 1pt}
}
\setlength{\emergencystretch}{3em}  % prevent overfull lines
\providecommand{\tightlist}{%
  \setlength{\itemsep}{0pt}\setlength{\parskip}{0pt}}
\setcounter{secnumdepth}{0}
% Redefines (sub)paragraphs to behave more like sections
\ifx\paragraph\undefined\else
\let\oldparagraph\paragraph
\renewcommand{\paragraph}[1]{\oldparagraph{#1}\mbox{}}
\fi
\ifx\subparagraph\undefined\else
\let\oldsubparagraph\subparagraph
\renewcommand{\subparagraph}[1]{\oldsubparagraph{#1}\mbox{}}
\fi

%%% Use protect on footnotes to avoid problems with footnotes in titles
\let\rmarkdownfootnote\footnote%
\def\footnote{\protect\rmarkdownfootnote}

%%% Change title format to be more compact
\usepackage{titling}

% Create subtitle command for use in maketitle
\providecommand{\subtitle}[1]{
  \posttitle{
    \begin{center}\large#1\end{center}
    }
}

\setlength{\droptitle}{-2em}

  \title{R-Practice}
    \pretitle{\vspace{\droptitle}\centering\huge}
  \posttitle{\par}
    \author{}
    \preauthor{}\postauthor{}
    \date{}
    \predate{}\postdate{}
  

\begin{document}
\maketitle

Practice:

\begin{enumerate}
\def\labelenumi{\arabic{enumi}.}
\tightlist
\item
  Create a 100 by 100 matrix where each row should be having 100
  elements having numbers from 101 to 200. All rows of the matrix will
  be same.
\end{enumerate}

101 102
103\ldots{}\ldots{}\ldots{}\ldots{}\ldots{}\ldots{}\ldots{}\ldots{}\ldots{}
200 101 102
103\ldots{}\ldots{}\ldots{}\ldots{}\ldots{}\ldots{}\ldots{}\ldots{}\ldots{}.200
101 102
103\ldots{}\ldots{}\ldots{}\ldots{}\ldots{}\ldots{}\ldots{}\ldots{}\ldots{}.200
\ldots{}.. \ldots{}.. \ldots{}.. \ldots{}.. 101 102
103\ldots{}\ldots{}\ldots{}\ldots{}\ldots{}\ldots{}\ldots{}\ldots{}\ldots{}200

\begin{Shaded}
\begin{Highlighting}[]
\NormalTok{m <-}\StringTok{ }\KeywordTok{matrix}\NormalTok{(}\KeywordTok{rep}\NormalTok{(}\DecValTok{101}\OperatorTok{:}\DecValTok{200}\NormalTok{, }\DataTypeTok{times =} \DecValTok{100}\NormalTok{),}\DecValTok{100}\NormalTok{, }\DecValTok{100}\NormalTok{, }\DataTypeTok{byrow =} \OtherTok{TRUE}\NormalTok{)}
\end{Highlighting}
\end{Shaded}

\begin{enumerate}
\def\labelenumi{\arabic{enumi}.}
\setcounter{enumi}{1}
\tightlist
\item
  Convert the matrix to data frame and give column names such that
  column numbers should be in string
\end{enumerate}

``1'' ``2'' ``3''
\ldots{}\ldots{}\ldots{}\ldots{}\ldots{}\ldots{}\ldots{}\ldots{}.``100''
101 102
103\ldots{}\ldots{}\ldots{}\ldots{}\ldots{}\ldots{}\ldots{}\ldots{}\ldots{}
200 101 102
103\ldots{}\ldots{}\ldots{}\ldots{}\ldots{}\ldots{}\ldots{}\ldots{}\ldots{}.200
101 102
103\ldots{}\ldots{}\ldots{}\ldots{}\ldots{}\ldots{}\ldots{}\ldots{}\ldots{}.200
\ldots{}.. \ldots{}.. \ldots{}.. \ldots{}.. 101 102
103\ldots{}\ldots{}\ldots{}\ldots{}\ldots{}\ldots{}\ldots{}\ldots{}\ldots{}200

\begin{Shaded}
\begin{Highlighting}[]
\NormalTok{df <-}\StringTok{ }\KeywordTok{as.data.frame}\NormalTok{(m)}
\KeywordTok{colnames}\NormalTok{(df) <-}\StringTok{ }\DecValTok{1}\OperatorTok{:}\KeywordTok{ncol}\NormalTok{(df)}
\end{Highlighting}
\end{Shaded}

\begin{enumerate}
\def\labelenumi{\arabic{enumi}.}
\setcounter{enumi}{2}
\tightlist
\item
  Add a column at ``50'' having all NA in a way such that the columns
  following 50 should be shifted accordingly.
\end{enumerate}

So the new data frame will have 101 columns and 100 rows

\begin{Shaded}
\begin{Highlighting}[]
\NormalTok{idx <-}\StringTok{ }\DecValTok{50}
\NormalTok{df <-}\StringTok{ }\KeywordTok{suppressWarnings}\NormalTok{(}\KeywordTok{cbind}\NormalTok{(df[}\DecValTok{1}\NormalTok{, }\DecValTok{1}\OperatorTok{:}\NormalTok{idx}\DecValTok{-1}\NormalTok{], }\OtherTok{NA}\NormalTok{, df[, idx}\OperatorTok{:}\KeywordTok{ncol}\NormalTok{(df)]))}
\KeywordTok{colnames}\NormalTok{(df) <-}\StringTok{ }\DecValTok{1}\OperatorTok{:}\KeywordTok{ncol}\NormalTok{(df)}
\CommentTok{# }\AlertTok{NOTE}\CommentTok{: For warning message refer: https://t.ly/28VE9}
\end{Highlighting}
\end{Shaded}

\begin{enumerate}
\def\labelenumi{\arabic{enumi}.}
\setcounter{enumi}{3}
\tightlist
\item
  Convert all the elements of the last 20 rows to 0 (zero)
\end{enumerate}

\begin{Shaded}
\begin{Highlighting}[]
\NormalTok{idx <-}\StringTok{ }\DecValTok{20}
\CommentTok{# }\AlertTok{NOTE}\CommentTok{: Operator ":" precedes "-", hence you need to put brackets}
\CommentTok{#       Refer: https://t.ly/60yqO}
\NormalTok{df[, (}\KeywordTok{ncol}\NormalTok{(df)}\OperatorTok{-}\NormalTok{idx)}\OperatorTok{:}\KeywordTok{ncol}\NormalTok{(df)] <-}\StringTok{ }\DecValTok{0} 
\end{Highlighting}
\end{Shaded}

\begin{enumerate}
\def\labelenumi{\arabic{enumi}.}
\setcounter{enumi}{4}
\tightlist
\item
  Create a function on your own that would take a vector of any number
  of elements, compute average, variance, standard deviation, range and
  create a dataframe having only one row but the columns should be
  labelled as ``average'', ``standard deviation'', ``variance'',
  ``range''
\end{enumerate}

\begin{Shaded}
\begin{Highlighting}[]
\NormalTok{make.stats.df <-}\StringTok{ }\ControlFlowTok{function}\NormalTok{(v, }\DataTypeTok{na.rm=}\NormalTok{F)\{}
\NormalTok{  my.range <-}\StringTok{ }\ControlFlowTok{function}\NormalTok{(v, }\DataTypeTok{na.rm=}\NormalTok{F)\{}
    \KeywordTok{max}\NormalTok{(v, }\DataTypeTok{na.rm=}\NormalTok{na.rm) }\OperatorTok{-}\StringTok{ }\KeywordTok{min}\NormalTok{(v, }\DataTypeTok{na.rm=}\NormalTok{na.rm)}
\NormalTok{  \}}
\NormalTok{  funcs <-}\StringTok{ }\KeywordTok{list}\NormalTok{(}\DataTypeTok{average=}\NormalTok{mean, }\StringTok{`}\DataTypeTok{standard deviation}\StringTok{`}\NormalTok{=sd, }
                \DataTypeTok{variance=}\NormalTok{var, }\DataTypeTok{range=}\NormalTok{my.range)}
\NormalTok{  call.func <-}\StringTok{ }\ControlFlowTok{function}\NormalTok{(func, v)\{}\KeywordTok{func}\NormalTok{(v, }\DataTypeTok{na.rm=}\NormalTok{na.rm)\}}
  
  \KeywordTok{return}\NormalTok{(}\KeywordTok{as.data.frame}\NormalTok{(}\KeywordTok{lapply}\NormalTok{(funcs, call.func, }\DataTypeTok{v=}\NormalTok{v), }\DataTypeTok{check.names=}\NormalTok{F))}
\NormalTok{\}}
\end{Highlighting}
\end{Shaded}

\begin{enumerate}
\def\labelenumi{\arabic{enumi}.}
\setcounter{enumi}{5}
\tightlist
\item
  For the same problem above, modify your function definition such that
  any zeroes in the vector are removed before the calculation is done.
  Just think that the zeros are some instrument error for the recordings
  that were done.
\end{enumerate}

\begin{Shaded}
\begin{Highlighting}[]
\NormalTok{make.stats.df2 <-}\StringTok{ }\ControlFlowTok{function}\NormalTok{(v, }\DataTypeTok{na.rm=}\NormalTok{F, }\DataTypeTok{zeroes.rm=}\NormalTok{T)\{}
  \ControlFlowTok{if}\NormalTok{ (zeroes.rm)\{}
\NormalTok{    v <-}\StringTok{ }\NormalTok{v[v }\OperatorTok{!=}\StringTok{ }\DecValTok{0}\NormalTok{]}
\NormalTok{  \}}
  \KeywordTok{make.stats.df}\NormalTok{(v, }\DataTypeTok{na.rm=}\NormalTok{na.rm)}
\NormalTok{\}}
\end{Highlighting}
\end{Shaded}

\begin{enumerate}
\def\labelenumi{\arabic{enumi}.}
\setcounter{enumi}{6}
\item
  Create a dataframe as

  ``a'' ``b'' ``c'' 1 1 2 3 2 1 4 9 3 1 8 27 4 1 16 81 
\end{enumerate}

Add elements row-wise and insert the addition result as a column

\begin{Shaded}
\begin{Highlighting}[]
\NormalTok{df <-}\StringTok{ }\KeywordTok{as.data.frame}\NormalTok{(}\KeywordTok{list}\NormalTok{(}\DataTypeTok{a=}\KeywordTok{rep}\NormalTok{(}\DecValTok{1}\NormalTok{, }\DecValTok{4}\NormalTok{), }\DataTypeTok{b=}\KeywordTok{sapply}\NormalTok{(}\DecValTok{1}\OperatorTok{:}\DecValTok{4}\NormalTok{, }\ControlFlowTok{function}\NormalTok{(x) }\DecValTok{2}\OperatorTok{^}\NormalTok{x),}
                         \DataTypeTok{c=}\KeywordTok{sapply}\NormalTok{(}\DecValTok{1}\OperatorTok{:}\DecValTok{4}\NormalTok{, }\ControlFlowTok{function}\NormalTok{(x) }\DecValTok{3}\OperatorTok{^}\NormalTok{x)))}
\NormalTok{df[}\StringTok{"add.result"}\NormalTok{] <-}\StringTok{ }\KeywordTok{rowSums}\NormalTok{(df) }\CommentTok{# df[a] + df[b] + df[c]}
\end{Highlighting}
\end{Shaded}

\begin{enumerate}
\def\labelenumi{\arabic{enumi}.}
\setcounter{enumi}{7}
\tightlist
\item
  For the same dataframe given above, add the column wise and insert the
  result as a row in the end
\end{enumerate}

\begin{Shaded}
\begin{Highlighting}[]
\NormalTok{df[}\StringTok{"add.result"}\NormalTok{] <-}\StringTok{ }\OtherTok{NULL} \CommentTok{# removing add.result column}
\NormalTok{df <-}\StringTok{ }\KeywordTok{rbind}\NormalTok{(df, }\KeywordTok{colSums}\NormalTok{(df))}
\KeywordTok{row.names}\NormalTok{(df)[}\KeywordTok{nrow}\NormalTok{(df)] <-}\StringTok{ "add.result"}
\end{Highlighting}
\end{Shaded}

\begin{enumerate}
\def\labelenumi{\arabic{enumi}.}
\setcounter{enumi}{8}
\tightlist
\item
  Create a function on your own that would take two 3-by-3 matrices and
  check whether for that pair AB is equal to BA or not
\end{enumerate}

\begin{Shaded}
\begin{Highlighting}[]
\NormalTok{commute <-}\StringTok{ }\ControlFlowTok{function}\NormalTok{(A, B)\{}
  \ControlFlowTok{if}\NormalTok{ (}\KeywordTok{all}\NormalTok{(A}\OperatorTok\NormalTok{B }\OperatorTok{==}\StringTok{ }\NormalTok{B}\OperatorTok\NormalTok{A))\{}
    \OtherTok{TRUE}
\NormalTok{  \}}\ControlFlowTok{else}\NormalTok{\{}
    \OtherTok{FALSE}
\NormalTok{  \}}
\NormalTok{\}}
\end{Highlighting}
\end{Shaded}

\begin{enumerate}
\def\labelenumi{\arabic{enumi}.}
\setcounter{enumi}{9}
\tightlist
\item
  Create a function which will take a 3x3 matrix as input and will
  return 3 vectors which are generated from the columns of the matrix.
\end{enumerate}

\begin{Shaded}
\begin{Highlighting}[]
\NormalTok{get.cols <-}\StringTok{ }\ControlFlowTok{function}\NormalTok{(A)\{}
\NormalTok{  l <-}\StringTok{ }\KeywordTok{lapply}\NormalTok{(}\DecValTok{1}\OperatorTok{:}\KeywordTok{ncol}\NormalTok{(A), }\ControlFlowTok{function}\NormalTok{(i) A[, i])}
  \KeywordTok{names}\NormalTok{(l) <-}\StringTok{ }\KeywordTok{colnames}\NormalTok{(A)}
\NormalTok{  l}
\NormalTok{\}}
\end{Highlighting}
\end{Shaded}


\end{document}
